\documentclass{article}
\usepackage{graphicx} % Required for inserting images

\title{General Physics Midterm 2023}
\author{teacher:Luyan Sun}
\date{November 14th 2023}

\begin{document}

\maketitle
\paragraph{Introduction}
This part doesn't actually appear on the exam paper, instead making a basic introduction to the details of the exam. The exam takes 3 hours. On the following recollection version for the exam contents, factual errors or typos may occur. Moreover, it's significantly challenging to recall the scores of each problem after exam, so we only write down the scores that we are certain with.
\section{(25pts)}
\paragraph{(1),5pts} State the uniqueness theorem. Include the boundary condition on conductor and dialectic surfaces.
\paragraph{(2)} Consider a dipole $p$ immersed in the environment of dielectric constant $\epsilon$. A grounded conducting sphere with radius $a$ is at distant $d$ to the dipole. Calculate the total electrostatic energy of the system.
\paragraph{(3)} A electron beam emitter is emitting electrons at speed $v_0$ and number density (with unit $m^{-3}$) $n$. The beam has radius $b$, and the electrons are uniformly distributed in the beam. A conducting sphere has center on the axis of the beam, and has radius $r>b$. The sphere is connected to the ground through resistance $R$. Let the absolute value of electron charge is $e$, now calculate the potential $V$ of the sphere when the system reaches equilibrium.
\paragraph{(4)} A spherical conductor has inner \textbf{diameter} $R_1$ and outer diameter $R_2$. Now, fill a kind of dielectric into it, such that the dielectric constant is $\epsilon=\epsilon_0+\epsilon_1 \cos^2\theta$. 
\begin{enumerate}
    \item Please calculate the capacitance $C$ of the system.
    \item Please calculate $\overrightarrow{E},\overrightarrow{P},\overrightarrow{D}$ inside the conductor and calculate the induced free charge on both inner and outer surfaces.
\end{enumerate}
\section{(15pts)}
\paragraph{(1),2pts} Consider two hemispheres shell with exactly opposite opening directions and same center position. The shells have radii $R,r$ respectively, and their charges are uniformly distributed on the surface, with total charge $Q,q$. For the special case $R=r,Q=q$, please calculate the total force exerted on $Q$.
\paragraph{(2),3pts} For the general case $R\not=r,Q\not=q,$ calculate again.
\paragraph{(3),5pts} Now, consider an irreverent configuration. A uniformly charged sphere has radius $R_1$ and total charge $Q$, its center is $O_1$. An arc has radius $R_2$, and its center $O_2$ is of distance $d$ to the center of the $R_1$ sphere. The charge on the arc surface is fixed, with distribution $\sigma(P)=\sigma_0 \cos(\theta_1-\theta_2)$, where $\theta_1=\angle PO_1 O_2$and $\theta_2=\pi -\angle P O_2 O_1$ . Now, please calculate the total force exerted on the arc.
\section{(10pts)}
\paragraph{(1)}Consider a fixed dipole moment $p=p\hat{z}$ at the origin. An charge with positive charge $q$ and mass $m$ is orbiting circularly with constant speed. Gravity is omitted at the moment. Please find the plane in which the charge is moving.
\paragraph{(2)} Please further calculate on the basis of (1), to find the velocity and total energy of the charge.
\paragraph{(3),2pts} If gravity can't be omitted, suppose the uniformly gravity field (or the gravitational acceleration) is $g$. Find the additional electric field to make the charge do the same movement as in (1)and (2).
\section{(10pts)}
Consider a electric quadruple on $xOy$ plane. It has four charges:$Q(-\frac{l}{2},\frac{l}{2}),-Q(\frac{l}{2},\frac{l}{2}),-Q(-\frac{l}{2},-\frac{l}{2}),Q(\frac{l}{2},-\frac{l}{2}).$
\paragraph{(1),2pts}Please calculate the electric potential at an arbitrary point $P(r,\theta)$ in the plane, where the polar axis is chosen to coincide with $x$ axis.
\paragraph{(2)}Please calculate the approximation result to first order when $r>>l$.
\paragraph{(3)}If an uniform electric field is exerted on the system, what is the interaction electric energy of the system?(By interaction we mean not taking the self-energy of the quadruple into account.)
\paragraph{(4)}If the electric field is not uniform, please re-calculate the energy in (3), keeping your result with the lowest order of $l$ as $l\rightarrow 0$.
\section{(10pts)}
Consider a coaxial cylinder conductor with height $H$, and inner and outer radius $a,b$, respectively. An voltage $V$ is kept fixed between the inner and outer surfaces. Now the conductor is placed in a sufficiently large pool of oil, whose electric permittivity is $\chi$ and density is $\rho$. In the problem, the gravity must be considered.
\paragraph{(1)} The oil would rise inside the conductor(i.e. in the place where $a<r<b$)  for a height $h<H$. At that time, find the total energy $W_{tot}$ of the system.
\paragraph{(2)} Please solve for the height $h$ when the system reaches equilibrium.
\section{(10pts)}
The Green's Reciprocal Theorem states that: if two charge systems have $(\Phi,\rho,\sigma)$ and $(\Phi^\prime,\rho^\prime,\sigma^\prime)$ respectively(they exist independently, but are in the same space), then:$$\int \Phi\rho^\prime dV+\int \Phi\sigma^\prime da=\int \Phi^\prime\rho dV+\int \Phi^\prime\sigma da$$

Use that fact to calculate the induced charge on two infinitely large parallel planes with distance $d$, with a point charge $q$ placing inside them and having distance $h$ from the first plane.
\section{(10pts)}
Please find the total energy for the following situations. You may write series in original form, without calculating the values of them.
\paragraph{(1)} An point charge $q$ is located at the middle of two infinitely large parallel grounded conducting surface, of which the distance is $l$.
\paragraph{(2)} An point charge $q$ is located at the middle of a infinity long conducting square tube of side length $l$.
\paragraph{(3)} An point charge $q$ is located at the middle of a grounded square conductor box of side length $l$.
\section{(10pts)}
\paragraph{(1)}Consider an interface, whose left side is dielectric constant $\epsilon_1$ and right side is $\epsilon_2$. A charge $q$ is placed exactly on the interface. Please calculate the $\overrightarrow{E},\overrightarrow{D}$ in the entire space, while also pointing out what is the polarized charge on the interface.
\paragraph{(2)} Consider the same interface but without the charge $q$ and with free charge density $\sigma$ instead. On a specific point on the interface, the electric field is given as $\overrightarrow{E_1}$at the left and $\overrightarrow{E_2}$ at the right. We denote the normal vector $\hat{n}$ to be pointing from 1 to 2. The angles between $\overrightarrow{E_1},\hat{n}$and $\overrightarrow{E_2},\hat{n}$ are $\theta_1,\theta_2$, respectively. Now, we know that $|\overrightarrow{E_1}|=E_0$, but don't know what is $|\overrightarrow{E_2}|$. Please give a formula for $\theta_1,\theta_2$.
\end{document}
